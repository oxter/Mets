В. Статический расчет поперечной рамы.
Расчет на постоянные нагрузки. Основная система приведена на рис., а схема нагрузки --- на рис.
Сосредоточенный момент из-за смещения осей верхней и нижней частей колонны
$$M = -(F_R + F_1)e_0=-(58.5+5.67)0.525=-33.7\,\text{кНм.}$$
По \cite[табл. 12.4]{veden} находим параметры $n=1/7=0.14\approx0.15$;
$$\alpha = H_\ruin{в}/H=4/13=0.3.$$
Каноническое уравнение имеет вид $$r_{11}\varphi+r_{1p}=0.$$
Моменты от поворота узлов на угол $\varphi = 1$ равны:
$$M_A=k_Ai=0.795i;\, M_C=k_Ci=-0.341i;\, M_B=k_Bi=-0.827i;$$
$$M_B^\ruin{р}=2EI_\ruin{р}/l=2E4I_\ruin{н}H/LH=8iH/l=8\cdot13i/30=3.47i.$$
Моменты от нагрузки на стойках $M_\ruin{р}$ равны:
$$M_A=k_AM=0.344\cdot(-33.7)=-11.6\,\text{кНм;}$$
$$M_B=k_BM=-0.159\cdot(-33.7)=5.4\,\text{кНм;}$$
$$M_C^\ruin{н}=k_AM=-0.708\cdot(-33.7)=23.9\,\text{кНм;}$$
$$M_C^\ruin{в}=(k_C+1)M=(-0.708+1)\cdot(-33.7)=-9.8\,\text{кНм.}$$
Моменты на опорах ригеля (защемляемая балка постоянного по длине сечения) 
$M_B^\ruin{р}=-q_gl^2/12=-3.9\cdot30^2/12=-293$ кНм.

Определение $r_{11}$ и $r_{1p}$:
$$r_{11}=M_B+M_B^\ruin{р}=0.827i+3.47i=4.3i\,(\text{по эпюре} M_1);$$
$$r_{1p}=M_B+M_B^\ruin{р}=-5.4-293=-298\,(\text{по эпюре} M_p).$$
Угол поворота $\varphi = - r{1p}/r{11}=-298/4.3i=69.3/i$.

Моменты от фактического угла поворота ($M_1\varphi$) равны:
$$M_A=0.795i\cdot69.3/i=55.1\,\text{кНм};\,M_B=-0.827i\cdot69.3/i=-57.3\,\text{кНм};$$
$$M_C=-0.341i\cdot69.3/i=-23.6\,\text{кНм};
\,M_B^\ruin{р}=3.47i\cdot69.3/i=240.5\,\text{кНм.}$$
Эпюра моментов ($M_{1\varphi}+M_p$) от постоянной нагрузки:
$$M_A=55.1-11.6=43.5\,\text{кНм};\,M_B=-57.3+5.4=-51.9\,\text{кНм};$$
$$M_C^\ruin{н}=23.9-23.6=0.3\,\text{кНм};\,M_C^\ruin{в}=-9.8-23.6=-33.4\,\text{кНм};$$
$$M_B^\ruin{р}=240.5-292=-52.5\,\text{кНм}.$$
Проверкой правильности расчета служит равенство моментов в узле B ($52.5\approx51.9$),
равенство перепада эпюры моментов в точке C ($33.4+0.3=33.7$) внешнему моменту (33.7),
а также равенство поперечных сил на верхней и нижней частях колонны:
$$Q_{AC}=-(43.5-0.3)/9=-4.8\,\text{кН};$$
$$Q_{BC}=-(51.9-33.4)/4=-4.63\,\text{кН}.$$
Разница (3.6\%) получена в результате округления параметра $n$.
На рис. приведена эпюра нормальных сил (с учетом веса стен и собственного веса колонн).

Расчет на нагрузку от снега.
Сосредоточенный момент на колонне 
$$M=F_r e_0=-189 \cdot 0.525=-99.2\knm.$$
Моменты от нагрузки:
$$M_A=0.344(-99.2)=-34.1\knm;\, M_B=-0.159(-99.2)=15.8\knm;$$
$$M_C^\ruin{н}=-0.708(-99.2)=70.2\knm;\, M_C^\ruin{в}=0.292(-99.2)=-29\knm;$$
$$M_B^\ruin{р}=-12.6\cdot30^2/12=-945\knm.$$
Определяем $r_{11}=4.3i$; $r_{1p}=-15.8-945=-960.6$.

Угол поворота $\varphi=960.8/4.3i=223.4/i$. Моменты от фактического угла поворота:
$$M_A=0.795i\cdot223.4/i=177.6\knm;\,M_B=-0.827i\cdot223.4/i=-184.8\knm;$$
$$M_C=-0.341i\cdot223.4/i=-76.2\knm;
\,M_B^\ruin{р}=3.47i\cdot223.4/i=775.2\knm.$$
Эпюры усилий от снеговой нагрузки показаны на рис.:
$$M_A=143.5\knm; \,M_B=-169\knm; \,M_C^\ruin{в}=-105.2\knm; \,M_C^\ruin{н}=-6\knm;$$
$$M_B^\ruin{р}=-169.8\knm; \, Q_B=-(169-105)/4=-16\,\text{кН};
\,Q_A=-(143.5+6)/9=-16.6\,\text{кН};$$
$$N_B=N_A=-189\,\text{кН}; \,N_\ruin{р}=-16.6\,\text{кН}.$$

Расчет на вертикальную нагрузку от мостовых кранов при расположении крана у левой стойки.
Основная система и схема нагрузки приведены на рис.

Проверку возможности считать ригель абсолютно жестким проводим по формуле
$$k=I_\ruin{р}H/I_]ruin{н}l=28 \cdot 13 / 7 \cdot 30 = 1.73.$$
$$1.73>6/(1+1.1\sqrt{I_\ruin{н}/I_\ruin{р}-1})=6.(1+1.1\sqrt{6})=1.62$$