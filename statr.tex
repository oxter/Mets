В. Статический расчет поперечной рамы.
Расчет на постоянные нагрузки. Основная система приведена на рис., а схема нагрузки --- на рис.
Сосредоточенный момент из-за смещения осей верхней и нижней частей колонны
$$M = -(F_R + F_1)e_0=-(58.5+5.67)0.525=-33.7\,\text{кНм.}$$
По \cite[табл. 12.4]{veden} находим параметры $n=1/7=0.14\approx0.15$;
$$\alpha = H_\ruin{в}/H=4/13=0.3.$$
Каноническое уравнение имеет вид $$r_{11}\varphi+r_{1p}=0.$$
Моменты от поворота узлов на угол $\varphi = 1$ равны:
$$M_A=k_Ai=0.795i;\, M_C=k_Ci=-0.341i;\, M_B=k_Bi=-0.827i;$$
$$M_B^\ruin{р}=2EI_\ruin{р}/l=2E4I_\ruin{н}H/LH=8iH/l=8\cdot13i/30=3.47i.$$
Моменты от нагрузки на стойках $M_\ruin{р}$ равны:
$$M_A=k_AM=0.344\cdot(-33.7)=-11.6\,\text{кНм;}$$
$$M_B=k_BM=-0.159\cdot(-33.7)=5.4\,\text{кНм;}$$
$$M_C^\ruin{н}=k_AM=-0.708\cdot(-33.7)=23.9\,\text{кНм;}$$
$$M_C^\ruin{в}=(k_C+1)M=(-0.708+1)\cdot(-33.7)=-9.8\,\text{кНм.}$$
Моменты на опорах ригеля (защемляемая балка постоянного по длине сечения) 
$M_B^\ruin{р}=-q_gl^2/12=-3.9\cdot30^2/12=-293$ кНм.

Определение $r_{11}$ и $r_{1p}$:
$$r_{11}=M_B+M_B^\ruin{р}=0.827i+3.47i=4.3i\,(\text{по эпюре} M_1);$$
$$r_{1p}=M_B+M_B^\ruin{р}=-5.4-293=-298\,(\text{по эпюре} M_p).$$
Угол поворота $\varphi = - r{1p}/r{11}=-298/4.3i=69.3/i$.

Моменты от фактического угла поворота ($M_1\varphi$) равны:
$$M_A=0.795i\cdot69.3/i=55.1\,\text{кНм};\,M_B=-0.827i\cdot69.3/i=-57.3\,\text{кНм};$$
$$M_C=-0.341i\cdot69.3/i=-23.6\,\text{кНм};
\,M_B^\ruin{р}=3.47i\cdot69.3/i=240.5\,\text{кНм.}$$