Расчет поперечной рамы

А. Расчетная схема рамы.

В соответствии с конструктивной схемой выбираем ее расчетную схему и основную систему. Расстояние между центрами тяжести верзнего и нижнего участков колонн
$$e_0=0.5(h_\ruin{н}-h_\ruin{в})=0.5\cdot(1750-700)=0.525\,\text{м}.$$
Соотношение моментов инерции $I_\ruin{н}/I_\ruin{в}=7; I_\ruin{р}/I_\ruin{н}=4$. Если $I_\ruin{в}=1$, то 
$I_\ruin{н}=5$. $I_\ruin{р}=20$. Сопряжение ригеля с колонной назначаем жестким (краны режима работы группы 8К, цех однопролетный).

Б. Нагрузки на поперечную раму.

Постоянная нагрузка. Нагрузка на 1 м$^2$ кровли определяем по \cite[табл. 17.3]{veden}. Расчет нагрузки в 
табл.~\ref{krov}.

\begin{table}[ht]
\caption{Постоянная нагрузка от покрытия}
\label{krov}
\centering
	\begin{tabular}{|l|p{2cm}|p{2cm}|p{2cm}|}
	\hline
		Состав покрытия & Нормативная нагрузка, кН/м$^2$ & 
		Коэффициент надежности по нагрузке & Расчетная нагрузка, кН/м$^2$ \\
	\hline
		Мембрана LOGICROOF V-RP & 0.02 & 1.3 & 0.026\\
		Мин. ватный утеплитель Техноруф В60 & 0.08 & 1.2 & 0.096\\
		Мин. ватный утеплитель Техноруф Н30 & 0.09 & 1.2 & 0.108\\
		Пароизоляция & 0.03 & 1.3 & 0.039\\
		Профилированный настил НС35-1000-0.55 & 0.06 & 1.05 & 0.063\\
		Собственный вес металлических конструкций & 0.3 & 1.05 & 0.315\\
	\hline
		& $g^\ruin{н}=0.58$ & & $g^\ruin{р}=0.65$\\
	\hline
	\end{tabular}
\end{table}

Расчетную равномерно распределенную линейную нагрузку на ригель рамы вычисляем по формуле
$$q_g=g_\ruin{кр}b_\ruin{ф}/\cos\alpha=0.65\cdot6/1=3.9\,\text{кН/м.}$$
Опорная реакция ригеля рамы $F_R=q_gl/2=3.9\cdot30/2=58.5$ кН.

Расчетный вес колонны.
По \cite[табл. 12.1]{veden} принято 0.3 кН/м$^2$. Вес верхней части (20\% веса) 
$G_\ruin{в}=1.05\cdot0.2\cdot0.3\cdot6\cdot15=5.67$ кН; вес нижней части (80\% веса)
$G_\ruin{н}=1.05\cdot0.8\cdot0.3\cdot6\cdot15=22.68$ кН.

Приняты самонесущие панели.

Снеговая нагрузка.
Вес снегового покрова $S_0=1.5$ кПа. Коэффициент надежности по нагрузке $\gamma_s=1.4.$
Линейная распределенная нагрузка от снега на ригель рамы равна
$$q_s=\gamma_sS_0b\ruin{ф}=1.4\cdot1.5\cdot6=12.6\,\text{кН/м.}$$
Опорная реакция ригеля $F_R=12.6\cdot30/2=189$ кН.

Вертикальные усилия от мостовых кранов см. на рис.
Базу крана (5.1~м), расстояние между колесами двух кранов (1.2~м), а также нормативное усилие колеса (345~кН) находим по \cite[прил.~1]{veden}. 
$$D_{max}=\gamma F \psi \Sigma F_k^n y + \gamma_g G_\ruin{пб}=
1.1 \cdot 0.95 \cdot 345 \cdot 1.9 + 1.05 \cdot 22.5 = 685 + 24 = 709\,\text{кН;}$$
(вес подкрановой балки по \cite[табл. 12.1]{veden} 
$G_\ruin{пб} = 0.25 \cdot 6 \cdot 15 = 22.5$ кН)
$${F'}_k=(Q+G_\ruin{кр})/n_0-F_k^n=(314+608)/2-345=116\,\text{кН;}$$
$$D_{min} = 685\cdot116/345+24=254\,\text{кН.}$$
Сосредоточенные моменты от вертикальных сил $D_{max}$, $D_{min}$ определяем по формуле
$$e_\ruin{к} = 0.5_\ruin{н} = 0.5 \cdot 1.75 = 0.875\,\text{м;} 
M_{max} = e\ruin{к} D_{max} = 0.875 \cdot 709 = 620\,\text{кНм;}$$
$$M_{min} = 0.875 \cdot 254 = 222\,\text{кНм.}$$
Горизонтальную силу от мостовых кранов находим по формулам
$$T_k^n = 0.05 (Q + G_\ruin{т}) / n_0 = 0.05 (314 + 85) / 2 = 10\,\text{кН;}$$
$$T = \gamma F \psi \Sigma T_k^n y = 1.1 \cdot 0.95 \cdot 10 \cdot 1.9 = 20\,\text{кН}$$
Считаем что сила $T$ приложена в уровне уступа колонны.

Ветровая нагрузка.
Нормативное давление ветра \cite[прил.~2]{veden} $w_0 = 0.3$ кПа. 
Тип местности Б \cite[прил.~3]{veden}, коэффициент $k$ при высоте до 5 м --- 0.5;
10 м --- 0.65 ; 20 м --- 0.85.

По формуле
$$q_w = \gamma_w w_0 k c b = 1.4 \cdot 0.3 \cdot 0.8 \cdot 6k = 2.016k.$$
Линейная распределенная нагрузка при высоте до 10 м равна $2.016 \cdot 0.65 = 1.31$
кН/м; 20 м --- $2.016 \cdot 0.85 = 1.71$ кН/м; 12.6 м --- 1.41 кН/м; 15.75 м --- 1.54 кН/м.

Сосредоточенные силы от ветровой нагрузки вычисляем по формулам:
$$F_w = (q_1 + q_2)h/2=(1.54+1.41)3.15/2=4.65\,\text{кН;}$$
$${F'}_w = F_w 0.6/0.8 = 3.49$$